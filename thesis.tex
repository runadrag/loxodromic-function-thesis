\documentclass[12pt,a4paper]{article}

\usepackage{fontspec}
\usepackage{polyglossia}
\usepackage[left=2.5cm,top=2.5cm,right=2.5cm,bottom=2.5cm,nohead]{geometry}
\usepackage{setspace}
\usepackage{listings} 
\usepackage{color}
\usepackage{float}
\usepackage{courier}
\usepackage{bold-extra}
\usepackage{fix-cm}
\usepackage{alltt}
\usepackage{indentfirst}
\usepackage{amsmath, amsthm, amssymb}
\usepackage{url}

 

\defaultfontfeatures{Mapping=tex-text}

\setmainfont{Liberation Serif}
\setsansfont{Liberation Sans}
\setmonofont{Liberation Mono}

\setmainlanguage{ukrainian}
\setotherlanguage{english}

\setstretch{1.1}
\begin{document}
\pretolerance=-1
\tolerance=2300

\pagenumbering{arabic}
\pagestyle{empty}
\setlength{\parindent}{1.5cm}
\fontsize{14pt}{6mm}\selectfont

\begin{center}
  Міністерство освіти і науки, молоді та спорту України
  
  Львівський національний університет імені Івана Франка

  Механіко-математичний факультет
\end{center}

\vspace{1cm}

\begin{flushright}
  Кафедра математичного і функціонального аналіза
\end{flushright}

\vspace{4cm}

\begin{center}
  {\bfseries\Large Властивості локсодромних функцій}
\end{center}

\vspace{2cm}

\begin{small}
\begin{flushleft}\leftskip8.5cm
  Магістерська робота студентки 
  групи МТМ-52м\\
  Московчук Наталі\linebreak
  
  Науковий керівник:\\
  професор\\
  Кондратюк А.А.
\end{flushleft}
\end{small}

\vspace{4cm}

\begin{center}
  Львів - 2013 
\end{center}

\clearpage


\setstretch{1.5}
\fontsize{14pt}{6mm}\selectfont

\tableofcontents
\clearpage
\pagestyle{plain}

\section{Вступ}

Теорія мероморфних мультиплікативно періодичних функцій була розроблена О.Раузенбергером. Ж.Валірон назвав ці функції локсодромними, бо точки, в яких такі функції у випадку не дійсного $q$ набувають однакові значення лежaть на логарифмічних спіралях. Образи цих точок на сфері Рімана перетинають меридіани під одним і тим же кутом, і називаються локсодромними кривими ($\lambda o \xi o\zeta$-косий, $\delta \rho o \mu o\zeta$-шлях). В полярних координатах вони прямі лінії.\\
Подвійно періодичні мероморфні функції є еліптичними функціями і є більш відомими з робіт К.Якобі, Н.Абеля, К.Вейєрштрасса.\\
Локсодромні мероморфні функції дають просту конструкцію еліптичних функцій.\\
В цій роботі вивчаються властивості локсодромних функцій.
\vspace{1,5cm}


\clearpage
\section{Допоміжні відомості}
Нехай $\mathbb{C}^{*}=\mathbb{C}\setminus\lbrace 0 \rbrace$\label{Cstar}
\newtheorem{ozn}{Означення}
\begin{ozn}
 Мероморфна функція $f$ в проколеній площині $\mathbb{C}^{*}$ називається \textbf{локсодромною}\label{loxodrFun} з мультиплікатором $q$, якщо задовільняється умова 
\begin{equation}\label{eq0}
\begin{array}{l}
f(qz)=f(z),\hspace{0,5cm} 0<\vert q \vert<1,\hspace{0,5cm} z\in \mathbb{C}^{*}
\end{array}
\end{equation}
\end{ozn}
\vspace{1,5cm}

\begin{ozn}
 Комплексна функція u+i v=f(x+i y) є \textbf{голоморфною} \label{holomFun}тоді і тільки тоді, коли виконуються умови Коші — Рімана \label{KoshiRimana}
\[\begin{array}{l}
    \frac{\partial u}{\partial x}=\frac{\partial v}{\partial y};\quad \frac{\partial u}{\partial y}=-\frac{\partial v}{\partial x}
\end{array}\]
й часткові похідні 
$\frac{\partial u}{\partial x},\,\frac{\partial u}{\partial y},\,\frac{\partial v}{\partial x},\,\frac{\partial v}{\partial y}$ - неперервні.
\end{ozn}
\vspace{1,5cm}

\begin{ozn}
  Позначимо
  \[\begin{array}{l}
   \rho(z)=\sum\limits^{+\infty}_{-\infty}  \frac{q^{n}z}{(1-q^{n}z)^2} ,
   \hspace{0,5cm} n\in \mathbb{Z}
   \end{array}\]
\end{ozn}
\vspace{1,5cm}

\newtheorem{pryk}{Приклад}
\begin{pryk}
Перевіримо, що функція $\rho(z)$ - локсодромна :\\
\[\begin{array}{l}
 \rho(qz)=\sum\limits^{+\infty}_{-\infty}  \frac{q^{n+1}z}{(1-q^{n+1}z)^2}=
\end{array}\]
зробивши заміну $n+1=m$, отримаємо
\[\begin{array}{l}
=\sum\limits^{+\infty}_{-\infty}  \frac{q^{m}z}{(1-q^{m}z)^2}=\rho(z)
\end{array}\]
\end{pryk}
\vspace{1cm}

\newtheorem{thm}{Теорема}
\newtheorem*{thmNoNum}{Теорема}
\begin{thmNoNum}
??????Зважаючи на теорему: голоморфна на компакті функція обмежена за модулем( бо неперервна за теоремою Вейєрштрасса)
\end{thmNoNum}
\vspace{1,5cm}

\begin{thmNoNum}
\textbf{(Ліувілля)}\label{thLiuv}
 Ціла функція, модуль якої обмежений, є сталою.
\end{thmNoNum}
\vspace{1,5cm}

\begin{ozn}
Якщо функція $f$ в деякому околі точки $z_{0}$ має неперервну похідну, то вона називається \textbf{аналітичною в точці} $\mathbf z_{0}$. \\
Якщо функція $f$ аналітична в кожній точці області $G$, то вона називається \textbf{аналітичною в} \label{analitFun}  $\mathbf G$.
\end{ozn}
\vspace{1,5cm}

\begin{ozn}
Нехай $\left [ C  \right ]$ позначає образ відрізка $\left [ a,b  \right ]$  при відображенні $z=z(t)$, тобто $ \left [ C  \right ]=\lbrace z:z=z(t), t \in \left [ a,b  \right ] \rbrace$. Крива називається \textbf{жордановою}\label{GordKryva}, якщо це відображення бієктивне????.\\
Якщо для $a < t_{1} < t_{2} < b$ виконується нерівність $$ z(t_{1})\neq z(t_{2}),$$  то крива називається \textbf{замкненою жордановою кривою}.\\
Зауважимо, що замкнена жордпнова крива не є жордановою кривою.
\end{ozn}
\vspace{1,5cm}

\begin{thmNoNum}
\textbf{(Коші про лишки)}\label{thKoshi}
 Нехай функція $f$ аналітична в замкненій області $\overline{G}$, обмеженій скінченним числом замкнених жорданових кривих, за винятком скінченної кількості ізольованих точок $z_{1}, z_{2},...,z_{q}\in \overline{G}$. Тоді
 \[\begin{array}{l}
\sum \limits_{ b_{j}\in C_{q}(R)} resf(z)= \frac{1}{2i\pi}\int \limits_{\partial C_{q}(R)} \frac{f(z)}{z}dz
\end{array} \]
\end{thmNoNum}
\vspace{1,5cm}

\begin{thmNoNum}
\textbf{(Принцип аргумента)}\label{prArg}
 Нехай область $G$ обмежена скінченним числом замкнених жорданових кривих, а функція $f$ аналітична в $\overline{G}$, за винятком скінченної кількості полюсів, причому на $\partial G$  функція не має ні нулів, ні полюсів. Тоді
\[\begin{array}{l}                    
N-P=\frac{1}{2\pi} \Delta_{\partial G} Arg f(z) =i\sum\limits_{z_k\in G} res \frac{f^{'}(z)}{f(z)} = \frac{1}{2i\pi}\int\limits_{\partial G}^{ }\frac{f^{'}(z)}{f(z)} dz 
\end{array}\]
де $N$ - кількість нулів, а $P$ - кількість полюсів функції $f$ в області $G$ з урахуванням їх порядків.
\end{thmNoNum}
\vspace{1,5cm}
\clearpage
\section{Властивості локсодромних функцій}

\newtheorem{twerd}{Твердження}
\begin{twerd}
 Локсодромні функції з мультиплікатором \textit{q} утворюють поле, яке позначається $L_{q}$ 
 \label{L_{q}}.
 \end{twerd} 
Поле — це комутативне кільце з одиницею, в якому кожний ненульовий елемент $a\neq 0$ має обернений $a^{-1}$ в цьому ж кільці.\\
Введемо операції додавання і множення.Вони вводяться поточково:\\
$f+g=h, \forall z \in \mathbb{C}^{*} h(z)=f(z)+g(z)\\
f*g=h, \forall z \in \mathbb{C}^{*} h(z)=f(z)*g(z)$\\
Нульвим елементом є $f(z)=0$ в $\mathbb{C}$.\\
Одиничним елементом є $f(z)=1$ в $\mathbb{C}$.\\
Тоді всі аксіоми поля будуть виконуватися:\\
1. Існування нульового елемента: $g(z)=0, f(qz)$-мероморфна, $f(qz)+g(z)=f(qz)$.\\
2. Додавання мероморфних функцій:$f_{1}(qz),f_{2}(qz)$-мероморфні і локсодромні, сума мероморфних- мероморфна, тому $f_{1}(qz)+f_{2}(qz)$-локсодромна \\
3. Частка. Потрібно показати наступне $f_{1}(qz)/f_{2}(qz)=F(qz)$\\
$f_{1}(qz)$, $f_{2}(qz)$ - мероморфні, частка двох мероморфних - мероморфна.Тому, оскільки $f_{1}(qz)$, $f_{2}(qz)$ -локсодромні, то $ F(qz)$ також локсодромна.\\
\emph{Зауважимо:} $f(z)$-локсодромна, $c \neq 0$. Тоді  $f(cqz)=f(q*cz)=f(cz)$
\vspace{1,5cm}

\newtheorem{zauv}{Зауваження}
\begin{zauv} 
Будь-яка функція 
\begin{equation}\label{eq1}
\begin{array}{l}
g(z)=f(e^{\frac{2iz\pi}{\omega_{1}}})
\end{array}
\end{equation}
є періодичною з періодом $\omega_{1}$.
\end{zauv}
Переконаємось, що це зауваження вірне: 
\[\begin{array}{l}
g(  z+\omega_{1}  ) = f(  e^{\frac{2i\pi (z+\omega_{1})}{\omega_{1}} }  ) =  f(  e^{\frac{2iz\pi }{ \omega_{1}}  +2i\pi } ) = f(  e^{\frac {2iz\pi}{\omega_{1}}} ) = g(z) 
\end{array}\]
\vspace{1cm}


\begin{twerd}
 Якщо функція $f$ - локсодромна з мультиплікатором $q$, то  
\[\begin{array}{l}
g(z)=f(e^{\frac{2iz\pi}{\omega_{1}}})
\end{array}\] 
є подвійно періодичною, a\hspace{0,5cm} $q=e^{\frac{2i\pi \omega_{2}}{ \omega_{1}}}$,\hspace{0,5cm} $q<1$, \hspace{0,5cm}  $Im(\omega_{2}/\omega_{1})>0$, тобто $g(z)$ еліптична функція.
\end{twerd}
\begin{proof}?????
\[\begin{array}{l}
\Lambda = \mathbb{Z}\omega_{1} + \mathbb{Z}\omega_{2} 
\end{array}\]

\[\begin{array}{l}
g(z+m\omega_{1}+ n\omega_{2})= f(e^{2i\pi(z+m\omega_{1}+ n\omega_{2})/\omega_{1}})=\\
=f( e^{2iz\pi}e^{2im\pi}e^{2in\pi\omega_{2} / \omega_{1}})=
f(e^{2iz\pi/\omega_{1}}q^{n})=
\end{array}\]
 оскільки $q$-мультиплікатор, то
 \[\begin{array}{l}
 =f(e^{2iz\pi/\omega_{1}})=g(z)
 \end{array}\]
\end{proof}
\vspace{1cm}

\begin{thm}
Якщо функція $f$ локсодромна і голоморфна в $\mathbb{C}^{*}$, то $f=const$.
\end{thm}

\includegraphics[scale=0.9]{th1.png}
\begin{proof}
Оскільки функція $f$ локсодромна, тому вона визначається своїми значеннями в замкненому кільці $\overline{C_{q}(1)}=\lbrace z\in \mathbb{C}: \vert q\vert\leq\vert z\vert\leq 1\rbrace$ \label{C_{q}(1)} тому, що 
\begin{equation}\label{Cond1}
\begin{array}{l}
 \forall z $ $\exists n: q^{n}z \in \overline{C_{q}(1)} 
\end{array}
\end{equation} 
Зауважимо, умова $\eqref{Cond1}$ рівносильна такій: $f(zq^{n})=f(z)$\\
Тобто ми отримали
\begin{equation}\label{Cond2}
\begin{array}{l}
\vert q \vert \leq \vert q\vert ^{n} \vert z\vert \leq 1                       
\end{array}
\end{equation}
$\overline{C_{q}(1)}$-обмежена і замкнена.\\
Щоб довести, що умова  $\eqref{Cond1}$ справді виконується, нам достатньо знайти $n$. З нерівності $\eqref{Cond2}$ отримаємо:
\[\begin{array}{l}
log\vert q \vert \leq n log\vert q\vert +log \vert z\vert \leq 0                       
\end{array}\]

\[\begin{array}{l}
\frac {log\vert q \vert }{log\vert q\vert} \geq n  + \frac{ log \vert z\vert}{log\vert q\vert} \geq 0                       
\end{array}\]

\[\begin{array}{l}
1- \frac{ log \vert z\vert}{log\vert q\vert} \geq n  \geq -\frac{ log \vert z\vert}{log\vert q\vert} 
\end{array}\]
Оскільки $\overline{C_{q}(1)}$-компакт, то функція $f$ компактна на $\overline{C_{q}(1)}$. Зважаючи на теорему:???? голоморфна на компакті функція обмежена за модулем( бо неперервна за теоремою Вейєрштрасса), отримуємо
\[\begin{array}{l}
  \vert f(z) \vert\leq M
\end{array}\]
Тобто наша функція $f$ компактна в $\mathbb{C}^{*}$. \\
А оскільки функція $f$ компактна в проколеній площині $\mathbb{C}^{*}$, то вона обмежена в околі нуля. Отже, $0$- усувна точка і по неперервності функція $f$ продовжується
\[\begin{array}{l}
  f(0)=\lim\limits_{z\rightarrow 0} f(z)
\end{array}\]
Тому 
\[\begin{array}{l}
  \vert f(z) \vert\leq M \Longrightarrow \vert f(0) \vert\leq M
\end{array}\]
Отож, функція $f$ ціла і обмежена в комплексній площині $\mathbb{C}$. Згідно з теоремою Ліувілля, ціла функція, модуль якої обмежений, є сталою.
\end{proof}
\vspace{1,5cm}

 
\begin{thm}
 Нехай функція $f\in L_{q}$ і не має ні нулів, ні полюсів на межі кільця 
\begin{equation}\label{th21}
\begin{array}{l}
C_{q}(R)=\lbrace z\in \mathbb{C}:\left |q  \right |R\leq \left |z  \right |<R \rbrace
\end{array}
\end{equation} \label{C_{q}(R)}
 Тоді сума лишків $\frac{f(z)}{z}$ дорівнює $0$.
\end{thm}
\begin{proof}
 Межа кільця $C_{q}(R)$ має вигляд $\lbrace z\in \mathbb{C}:\left |q  \right |R< \left |z  \right |<R \rbrace$, адже межею є відкрита множина, але задля спрощення записів, розглядатимемо суму лишків на $\lbrace z\in \mathbb{C}:\left |q  \right |R\leq \left |z  \right |<R \rbrace$, маючи на увазі його межу. Згідно з теоремою Коші про лишки
 \[\begin{array}{l}
\sum \limits_{ b_{j}\in C_{q}(R)} resf(z)= \frac{1}{2i\pi}\int \limits_{\partial C_{q}(R)} \frac{f(z)}{z}dz
\end{array} \]
Нехай $\Gamma$, $\Gamma^{'}$ позначають кола, які обмежують кільце $C_{q}(R)$.\\
\includegraphics[scale=0.5]{1..PNG}\\
Тоді 
\[\begin{array}{l}
\sum_{b_{j} \in C_{q}(R)}^{ } resf(z) = 
\end{array} \]
Оскільки правильним напрямком обходу області інтегрування називається такий
напрямок, коли область інтегрування залишається ліворуч, тому
\[\begin{array}{l}
=\frac{1}{2\pi}(\int\limits_{\Gamma^{'}_{+}} \frac{f(\xi)}{\xi}d\xi + \int\limits_{\Gamma_{-}}\frac{f(x)}{x})dx)=
\end{array} \]
А оскільки інтеграл по правильному напрямку обходу області інтегрування рівний зі знаком "мінус" інтегралу по протилежному напрямку обходу області інтегрування, тобто $\int\limits_{\Gamma_{-}}= - \int\limits_{\Gamma_{+}}$, тому 
\[\begin{array}{l}
=\frac{1}{2\pi}(\int\limits_{\Gamma^{'}_{+}} \frac{f(\xi)}{\xi}d\xi - \int\limits_{\Gamma_{+}}\frac{f(x)}{x})dx)=
\end{array} \]
 Записавши межі інтегрування в параметричному вигляді, отримаємо 
 \[\begin{array}{l}
 = \frac{1}{2i\pi}(\int\limits_{0}^{2\pi }f(Re^{i\theta})\frac{iRe^{i\theta}}{Re^{i\theta }}d\theta  + \int\limits_{0}^{2\pi }f(\left |q \right |Re^{i\theta })\frac{i\left |q \right |Re^{i\theta }}{\left |q \right |Re^{i\theta }}d\theta)=
\end{array} \] 
\[\begin{array}{l}
= \frac{1}{2\pi}(\int\limits_{0}^{2\pi} f(Re^{i\theta})d\theta  - \int\limits_{0}^{2\pi}f(\left |q \right |Re^{i\theta })d\theta) = 
\end{array} \]
Зробимо наступні заміни змінних:
\begin{equation}\label{th22}
\begin{array}{l}
\theta =\varphi +\alpha 
\end{array} 
\end{equation}
\[\begin{array}{l}
q=\left |q  \right |e^{i\alpha }
\end{array} \]
і отримаємо
\[\begin{array}{l}
= \frac{1}{2\pi}(\int\limits_{0}^{2\pi} f(Re^{i\theta})d\theta  - \int\limits_{-\alpha}^{2\pi-\alpha}f(qRe^{i\varphi })d\varphi)=
\end{array} \]
А змінивши межі інтегрування другого інтеграла\\
 $$2\pi-\alpha\longrightarrow 2\pi ,$$ 
 $$ -\alpha\longrightarrow 0 ,$$
  $$f(Re^{i\varphi}) \longrightarrow f(Re^{i\theta}) ,$$ згідно з заміною $\eqref {th22} $, одержимо
\[\begin{array}{l}
=\frac{1}{2\pi}(\int\limits_{0}^{2\pi} f(Re^{i\theta})d\theta - \int\limits_{0}^{2\pi}f(Re^{i\theta })d\theta)=0,
\end{array}\]
що і потрібно було довести.
\end{proof}
\vspace{1,5cm}

\newtheorem{nasl}{Наслідок}
\begin{nasl}
 Кожна не стала локсодромна функція з мультиплікатором $q$ має щонайменше 2 полюси( і 2 нулі ) в кожному кільці $C_{q}(R)$.
\end{nasl}
\begin{proof}
Те, що функція має щонайменше 2 полюси означає, що полюсів може бути два і більше. Доводимо від супротивного- полюсів $0$ або $1$. $ f \neq const $\\
Якщо нема жодного полюса, то функція $f$ голоморфна і стала за теоремою 1, що суперечить припущенню нашого наслідку. \\
Припустимо, що наша локсодромна функція має один полюс.Тоді за теоремою Коші про лишки

\begin{equation}\label{nasl1}
\begin{array}{l}
\frac{1}{2i\pi}\int\limits_{\partial C_{q}(R)}^{ } \frac{f(z)}{z}dz= resf(z)\big\vert_{z=b} \neq 0 
\end{array} 
\end{equation}
бо $0$ тоді і лише тоді, коли функція $f$ голоморфна. Рівність
$\eqref {nasl1} $ суперечить твердженню теореми 2. \\
Тобто, справді кожна не стала локсодромна функція має щонайменше 2 полюси в будь-якому кільці $C_{q}(R)$.
\end{proof}
\vspace{1,5cm}

\begin{thm}
Нехай функція $f\in L_{q}$, $C_{q}(R)$-кільце, таке як в теоремі 2. І нехай $m_{a}(n_{b})$ позначають кількість нулів(полюсів) функції $f \in C_{q}(R)$. Тоді 
\begin{center}
$\sum m_{a}=\sum n_{b}$
\end{center}
\end{thm}
\begin{proof}
Нагадаємо, що запис принципу аргумента виглядає так:
\begin{equation}\label{arg}
\begin{array}{l}                    
N-P=\frac{1}{2\pi} \Delta_{\partial G} Arg f(z) =i\sum\limits_{z_k\in G} res \frac{f^{'}(z)}{f(z)} = \frac{1}{2i\pi}\int\limits_{\partial G}^{ }\frac{f^{'}(z)}{f(z)} dz , 
\end{array}
\end{equation}
де $N$ - кількість нулів, а $P$ - кількість полюсів функції $f$ з урахуванням їх порядків.
Застосуємо цей принцип до різниці суми нулів і суми полюсів в нашому випадку $\eqref {arg} $
\[\begin{array}{l}
\sum m_{a}-\sum n_{b} = \frac{1}{2i\pi}(\int\limits_{\Gamma^{'}_{+}}^{ } \frac{f^{'}(\xi )}{f(\xi) }d\xi-\int\limits_{\Gamma_{+}}^{ } \frac{f^{'}(z)}{f(z)}dz)=
\end{array}\]
запишемо межі інтегрування в параметричному вигляді
\[\begin{array}{l}
= \frac{1}{2i\pi}(\int\limits_{0}^{2\pi }iqRe^{i\varphi}\frac{f^{'}(qRe^{\varphi})}{f(qRe^{i\varphi })}d\varphi -
    \int\limits_{0}^{2\pi }iRe^{i\varphi}\frac{f^{'}(Re^{\varphi})}{f(Re^{i\varphi })}d\varphi)=
\end{array}\]
\[\begin{array}{l}
= \frac{1}{2i\pi}(\int\limits_{0}^{2\pi }\frac{1}{q}iqRe^{i\varphi}\frac{f^{'}(Re^{\varphi})}{f(Re^{i\varphi })}d\varphi -
    \int\limits_{0}^{2\pi }iRe^{i\varphi}\frac{f^{'}(Re^{\varphi})}{f(Re^{i\varphi })}d\varphi)=0.
\end{array}\]
Отже, ми довели, що сума нулів функції рівна сумі її полюсів.
\end{proof}
\vspace{1,5cm}

\begin{twerd} Для $z \in \mathbb{Z}^{*}=\mathbb{Z}\setminus \lbrace 0 \rbrace$\label{{Z}^{*}} покладемо
\[\begin{array}{l} 
 S(z)= \prod\limits_{0}^{+\infty } ( 1-q^{n}z ) \prod\limits_{1}^{+\infty }( 1-\frac{q^{n}}{z})
 \end{array}\]\label{S(z)}
Тоді
\begin{equation}\label{s1}
\begin{array}{l}
   S(qz)=-\frac{1}{z}S(z)
 \end{array}
\end{equation}  
\begin{equation}\label{s2}
\begin{array}{l}        
   S(\frac{1}{z})= -\frac{1}{z}S(z)
\end{array}
\end{equation}
\end{twerd}
\begin{proof}
\[\begin{array}{l} 
S(qz)= \prod\limits_{0}^{+\infty } ( 1-q^{n+1}z ) \prod\limits_{1}^{+\infty }( 1-\frac{q^{n}}{qz})=
 \prod\limits_{0}^{+\infty } ( 1-qq^{n}z ) \prod\limits_{1}^{+\infty }( 1-\frac{1}{q}\frac{q^{n}}{z})=
 \end{array}\]
зробимо заміни: для першого добутку $n+1=m$, для другого $n-1=m$, тоді 
\[\begin{array}{l} 
 =\prod\limits_{1}^{+\infty } ( 1-q^{m}z ) \prod\limits_{0}^{+\infty }( 1-\frac{q^{m}}{z})=\frac{\prod\limits_{0}^{+\infty } ( 1-q^{m}z )}{1-z}(1-\frac{1}{z})\prod\limits_{1}^{+\infty }( 1-\frac{q^{m}}{z})=-z^{-1}S(z)
\end{array}\]
\[\begin{array}{l} 
 S(\frac{1}{z})= \prod\limits_{0}^{+\infty } ( 1-q^{n}\frac{1}{z} ) \prod\limits_{1}^{+\infty }( 1-q^{n}z)=
\end{array}\]
домножимо перший добуток на $(1-q^{0}\frac{1}{z})$, а другий на $(1-z)$, тоді 
\[\begin{array}{l} 
  =\prod\limits_{1}^{+\infty } ( 1-q^{n}\frac{1}{z} )(1-q^{0}\frac{1}{z})\frac{    \prod\limits_{0}^{+\infty }( 1-q^{n}z)}{1-z}=-z^{-1}S(z)
 \end{array}\]
\end{proof}
\vspace{1,5cm}

Тепер розглянемо $a_{1},...,a_{m},b_{1},...,b_{m}\in \mathbb{C}^{*}$ такі що\\
\begin{equation}\label{w1}
\begin{array}{l} 
 a_{i}\neq b_{j} \hspace{0,5cm}  i  \hspace{0,5cm} a_{1}*...*a_{m}=b_{1}*...*b_{m}
\end{array}
\end{equation} 

Якщо для $z \in \mathbb{C}^{*}$ не рівному $b_{1},...,b_{m}$ по модулю $q $, ми покладемо
\begin{center}
\[ M(z)=\frac{S(\frac{z}{a_{1}})...S(\frac{z}{a_{m}})}{S(\frac{z}{b_{1}})...S(\frac{z}{b_{m}})}\]\label{M(z)}
\end{center}
тоді бачимо, що $M$ мероморфна функція на $\mathbb{C}^{*}$, її полюси $b_{1},...,b_{m}$ конгруентні по модулю $q$. Більше того, з $\eqref{s1}$ і $\eqref{s2}$ випливає

\begin{equation}\label{q2}
\begin{array}{l} 
 M(qz)=M(z), \hspace{0,5cm} M(\frac{1}{z})=\frac{S(a_{1}z)...S(a_{m}z)}{S(b_{1}z)...S(b_{m}z)}     
\end{array}
\end{equation}  
 
тобто у випадку виконання умови  $\eqref{w1}$, ми бачимо, що $M\in L_{q}$.\\
Перевіримо $\eqref{q2}$:\\  
 \[\begin{array}{l}
 M(qz)=\frac{S(\frac{qz}{a_{1}})...S(\frac{qz}{a_{m}})}{S(\frac{qz}{b_{1}})...S(\frac{qz}{b_{m}})}=
 \end{array}\]
 застосуємо рівність $\eqref{s1}$
\[\begin{array}{l}
=\frac{ (-1)^{m} \frac{a_{1}}{z} S(\frac{z}{a_{1}})...\frac{a_{m}}{z} S(\frac{z}{a_{m}}) }{ (-1)^{m} \frac{b_{1}}{z} S(\frac{z}{b_{1}})...\frac{b_{m}}{z} S(\frac{z}{b_{m}}) }

=\frac{ a_{1}*...*a_{m}}{b_{1}*...*b_{m}}\frac{S(\frac{z}{a_{1}})...S(\frac{z}{a_{m}})}{S(\frac{z}{b_{1}})...S(\frac{z}{b_{m}})}=
\end{array}\]
\[\begin{array}{l}
=\frac{ a_{1}*...*a_{m}}{b_{1}*...*b_{m}}M(z) 
\end{array}\]
зважаючи на $\eqref{w1}$, отримаємо
$M(qz)=M(z)$\\

Аналогічно застосувавши $\eqref{s2}$ і $\eqref{w1}$, отримаємо\\
\[\begin{array}{l}
M(\frac{1}{z})=\frac{S(\frac{1}{za_{1}})...S(\frac{1}{za_{m}})}{S(\frac{1}{zb_{1}})...S(\frac{1}{zb_{m}})}= 
\frac{(-1)^m \frac{1}{za_{1}} S(za_{1})...\frac{1}{za_{m}} S(za_{m})}{(-1)^m \frac{1}{zb_{1}} S(zb_{1})...\frac{1}{zb_{m}} S(zb_{m})}=
\end{array}\]
\[\begin{array}{l}
= \frac{b_{1}...b_{m}}{a_{1}...a_{m}}
\frac{ S(za_{1})...S(za_{m})}{S(zb_{1})...S(zb_{m})}=\frac{ S(za_{1})...S(za_{m})}{S(zb_{1})...S(zb_{m})}  
\end{array} \]
\vspace{1,5cm}

\begin{thm}
Для кожного $R$, такого що, межа $C_{q}(R)$, яка не містить ні нулів, ні полюсів відмінної від сталої локсодромної функції $f$, нехай $\lambda=\frac{(a_{1}...a_{m})}{(b_{1}...b_{m})}$ , де $a_{1},...,a_{m}$- нулі функції $f$, $b_{1},...,b_{m}$- її полюси. Тоді $\lambda$ належить циклічній групі $\left \langle q \right \rangle$, тобто $ \exists n:\lambda =q^{n}$.
\end{thm}
\begin{proof}
Першу рівність з $\eqref{q2}$ перепишемо:
\[\begin{array}{l}
 M(qz)=\lambda M(z).
 \end{array}\]
 Розглянемо функцію:
 \[\begin{array}{l}
 g(z)=\frac{f(z)}{M(z)}.
  \end{array}\]
  Згідно побудови функції $M$,бачимо, що функція $g$ не має ні нулів, ні полюсів в $C_{q}(R)$, бо $M$ має ті ж нулі і полюси, що й функція $f$. Звідси слідує, що $g$ ціла??????\\
   ???????\\
   так само, як $\frac{1}{g}$.
\begin{equation}\label{th41}
\begin{array}{l}
g(qz)=\frac{f(qz)}{M(qz)}=\frac{f(z)}{\lambda M(z)}=\frac{1}{\lambda}g(z)
\end{array}
\end{equation}
Нехай $\sum\limits_{-\infty}^{+\infty}c_{n}z^{n}$-розвинення в ряд Лорана функції $g$ в $\mathbb{C}^{*}.\\$
 \[\begin{array}{l}
g(z)=\sum\limits_{-\infty}^{+\infty}c_{n}z^{n},
 \end{array}\]
 \[\begin{array}{l}
 g(qz)=\frac{1}{\lambda}g(z)
 \end{array}\]
 \[\begin{array}{l}
 \sum\limits_{-\infty}^{+\infty}c_{n}q^{n}z^{n}=\frac{1}{\lambda}\sum\limits_{-\infty}^{+\infty}c_{n}z^{n}
  \end{array}\]
 \[\begin{array}{l} 
c_{n}q^{n}=\frac{1}{\lambda}c_{n} 
  \end{array}\] 
   \[\begin{array}{l}
 (\lambda q^{n}-1)c_{n}=0,
  \end{array}\]
Хоча б один з коефіцієнтів $c_{n}$ відмінний від нуля, бо якщо б всі $c_{n}=0$, то $g(z)=c_{0}$ і вона була б стала, що суперечило б умові.Отже 
$$\exists k: c_{k}\neq 0$$ 
і 
 \[\begin{array}{l} 
 \lambda= q^{-n}\in \left \langle q \right \rangle .
 \end{array}\]
Розвинення в ряд Лорана єдине, тому коефіцієнти при відповідних степенях рівні.
\[\begin{array}{l}
c_{k}q^{k}=\frac{1}{\lambda} c_{k}$ $\Rightarrow$ $(q^{k}\lambda -1)c_{k}=0,
 \end{array}\]
\[\begin{array}{l}
c_{k}\neq 0$ $\Rightarrow$  $\lambda=q^{-k}.
 \end{array}\]
??????Для $g(z)$ лише одне $ c_{k}\neq 0$. ????
\end{proof}
\vspace{1,5cm}

\begin{thm}
Якщо $\lambda =q^{n}$, $f$- локсодромна функція з теореми 4, то  $f$ має вигляд
\begin{equation}\label{th51}
\begin{array}{l}
 f(z)= C\tfrac{S(\frac{z}{a_{1}})...S(\frac{z}{a_{m}})}{S(\frac{zq^{n}}{b_{1}})S(\frac{z}{b_{2}})...S(\frac{z}{b_{m}})}
\end{array}
\end{equation}
 \end{thm}
\begin{proof}
\[\begin{array}{l}
g(z)=\frac{f(z)}{M(z)}\Rightarrow f(z)=M(z)g(z)= M(z)z^{n}c_{n}
 \end{array}\]
 позначивши $c_{n}=C$, отримаємо 
\[\begin{array}{l} 
f(z) =Cz^{n}\frac{S(\frac{z}{a_{1}})...S(\frac{z}{a_{m}})}{S(\frac{z}{b_{1}})...S(\frac{z}{b_{m}})} 
\end{array}\]
\[\begin{array}{l}
f(z)=C\tfrac{S(\frac{z}{a_{1}})...S(\frac{z}{a_{m}})}{S(\frac{zq^{n}}{b_{1}})S(\frac{z}{b_{2}})...S(\frac{z}{b_{m}})}= 
 \end{array}\]
оскільки
\[\begin{array}{l}
S(\frac{zq^{n}}{b_{1}})= -\frac{b_{1}}{z}S(\frac{zq^{n-1}}{b_{1}})=(-1)^{n}\frac{b_{1}^{n}}{z^{n}}S(\frac{z}{b_{1}}).
 \end{array}\]
тому
\[\begin{array}{l}
 =C\tfrac{S(\frac{z}{a_{1}})...S(\frac{z}{a_{m}})}{(-1)^{n}\frac{b_{1}^{n}}{z^{n}} S(\frac{z}{b_{1}})S(\frac{z}{b_{2}})...S(\frac{z}{b_{m}})}=Cz^{n}\frac{S(\frac{z}{a_{1}})...S(\frac{z}{a_{m}})}{S(\frac{z}{b_{1}})...S(\frac{z}{b_{m}})} 
  \end{array}\]
Ми отримали $\eqref{th51}$. При чому константа $C$ дорівнює $(-1)^{n}b_{1}^{-n}$.
\end{proof}


\clearpage
\section{Приклади}


\begin{pryk}
\textbf{Завдання:} Виразити нулі і полюси
\[\begin{array}{l}
g(z)=f(e^{\frac{2\pi i \omega_{2} }{\omega_{1}}})
\end{array}\]
через нулі і полюси локсодромної функції $f$ з мультиплікатором 
\[\begin{array}{l}
q=e^{\frac{2\pi i \omega_{2} }{\omega_{1}}}
\end{array}\]
\textbf{Розв'язок:} Нехай $\alpha_{k} $-послідовність всіх нулів функції $f(z)$, а $a_{k}$-послідовність всіх нулів функції $g(z)$, тоді
\begin{equation}\label{pr2}
\begin{array}{l}
\alpha_{k}=e^{\frac{2\pi i \omega_{2} }{\omega_{1}}}
\end{array}
\end{equation}
Прологарифнуємо $\eqref{pr2}$, враховуючи, що логарифм-функція багатозначна, отримаємо
\begin{equation}\label{pr22}
\begin{array}{l}
log(\alpha_{k})=log(e^{\frac{2\pi i \omega_{2} }{\omega_{1}}})= \frac{2\pi i \omega_{2} }{\omega_{1}}+ 2\pi m_{1} 
\end{array}
\end{equation}
Виразивши з $\eqref{pr22}$ 
\[\begin{array}{l}
a_{k}=\frac{w_{1}}{2\pi i}log\alpha_{k} + m_{1}i\omega_{1}=\frac{w_{1}}{2\pi i}(log\vert\alpha_{k}\vert + iarg\alpha_{k})+m_{1}i\omega_{1}
\end{array}\]
$a_{k}$-нулі функції $g(z)$.
\end{pryk}
\vspace{1,5cm}

\begin{pryk}
\textbf{Завдання:} Довести, що будь яка логарифмічна спіраль перетинає всі промені під одним і тим же кутом.\\

\textbf{Розвязок:} Нехай $z=re^{i\phi}$ - проста логарифмічна спіраль і нехай $r=e^{\phi}$. 
\end{pryk}
\vspace{1,5cm}


І так далі\cite{web}


\clearpage
\section{Список основних позначень}
$\mathbb{C}^{*}=\mathbb{C}\setminus\lbrace 0 \rbrace$, ~\pageref{Cstar}\\
$L_{q}$, ~\pageref{L_{q}}\\
$\overline{C_{q}(1)}=\lbrace z\in \mathbb{C}: \vert q\vert\leq\vert z\vert\leq 1\rbrace$,  ~\pageref{C_{q}(1)}\\
$C_{q}(R)=\lbrace z\in \mathbb{C}:\left |q  \right |R\leq \left |z  \right |<R \rbrace$, ~\pageref{C_{q}(R)}\\
$\mathbb{Z}^{*}=\mathbb{Z}\setminus \lbrace 0 \rbrace$,  ~\pageref{{Z}^{*}} \\
$S(z)= \prod\limits_{0}^{+\infty } ( 1-q^{n}z ) \prod\limits_{1}^{+\infty }( 1-\frac{q^{n}}{z})$, 
~\pageref{S(z)}\\
$ M(z)=\frac{S(\frac{z}{a_{1}})...S(\frac{z}{a_{m}})}{S(\frac{z}{b_{1}})...S(\frac{z}{b_{m}})}$, ~\pageref{M(z)}



\section{Предметний покажчик}
Теорема Коші, ~\pageref{thKoshi}\\
Теорема Ліувілля, ~\pageref{thLiuv}\\
Принцип аргументу, ~\pageref{prArg}\\
Жорданова крива, ~\pageref{GordKryva}\\
Аналітична функція, ~\pageref{analitFun}\\
Локсодромна функція, ~\pageref{loxodrFun}\\
Голоморфна функція, ~\pageref{holomFun}\\
Умови Коші-Рімана, ~\pageref{KoshiRimana}\\



\clearpage
\section{Анотація}

У першому розділі роботи подано основні означення, приклади
еліптичних та локсодромних функцій, основні властивості
локсодромних функцій у вигляді теорем.

У другому розділі показано як пов'язані між собою локсодромні та
еліптичні функції, виражаються нулі еліптичної функції через нулі
та полюси локсодромної функції.

У третьому розділі розглядається локсодромна функція $
\rho(z)=\frac{q^{n}z}{(1-q^{n}z)^{2}}$, за допомогою цією функції
будується функція $
\rho_{1}(z)=\frac{q^{n}z}{(1-q^{n}z)^{2}}+\frac{1}{12}-2\sum_{n=1}^{+\infty}\frac{nq^{n}}{1-q^{n}}$,
доводиться що локсодромна функція $ \rho_{1}$ задовільняє таке
диференціальне
рівняння:$[z\rho_{1}'(z)]^{2}=4\rho_{1}^{3}-g_{4}\rho_{1}-g_{6} $.

У четвертому розділі роботи розглядається функція Вейєрштрасса
$\wp(z)=\frac{1}{z^{2}}+\sum_{w}(\frac{1}{(z-w)^{2}}-\frac{1}{w^{2}}),
де w=2m\omega_{1}+2n\omega_{2}\neq0 $, похідна цієї функції,
доводиться її парність та еліптичність.

У п'ятому розділі розглядається зв'язок між функціями $\rho_{1}$
та $\wp$, за допомогою цього зв'язку отримуються додаткові
властивості функції Вейєрштрасса.
\clearpage

\addcontentsline{toc}{section}{Використана література}
\renewcommand\refname{Використана література}
\begin{thebibliography}{9}

  \bibitem{alias}Yves Hellegouarch \emph{Invitation to the mathematics of Fermat-Wiles},
    2001, 385 ст.

  \bibitem{alias}Валірон \emph{ooooo},
    0000, 000 ст.
    
   \bibitem{alias}A.Kondratyuk, I.Laine \emph{Meromorphic functions in multiply connected domains},
    Joensuu-L'viv 2006, 116 ст.
    
  \bibitem{alias}Шабат Б.В. \emph{Введение в комплексный анализ. Часть 1}, http://www.huminst.ru/lib/%D0%9C%D0%B0%D1%82%D0%B5%D0%BC%D0%B0%D1%82%D0%B8%D0%BA%D0%B0/%D0%A8%D0%B0%D0%B1%D0%B0%D1%82%20%D0%91.%D0%92.%20-%20%D0%92%D0%B2%D0%B5%D0%B4%D0%B5%D0%BD%D0%B8%D0%B5%20%D0%B2%20%D0%BA%D0%BE%D0%BC%D0%BF%D0%BB%D0%B5%D0%BA%D1%81%D0%BD%D1%8B%D0%B9%20%D0%B0%D0%BD%D0%B0%D0%BB%D0%B8%D0%B7.pdf
    , Москва "Наука" 1985, 335 ст.    

 \bibitem{alias}Шабат Б.В. \emph{Введение в комплексный анализ.Часть 2 },
    Москва "Наука" 1985, 463 ст. 
        
  \bibitem{alias}Гольдберг А.А., Шеремета М.М. \emph{Аналітичні функції},
    Київ НМК ВО 1991, 115 ст.

  \bibitem{alias}Гольдберг А.А., Шеремета М.М.,Заболоцький М.В.,Сказків О.Б \emph{Комплексний аналіз},
    Львів Афіша 2008, 203 ст.
          
  \bibitem{web}<АВТОР> \emph{<НАЗВА>} [Електронний ресурс],
    <РІК>. Режим доступу:
    \url{https://github.com/Uko/thesis-template}

\end{thebibliography}

\end{document}
