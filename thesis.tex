\documentclass[12pt,a4paper]{article}

\usepackage{fontspec}
\usepackage{polyglossia}
\usepackage[left=2.5cm,top=2.5cm,right=2.5cm,bottom=2.5cm,nohead]{geometry}
\usepackage{setspace}
\usepackage{listings} 
\usepackage{color}
\usepackage{float}
\usepackage{courier}
\usepackage{bold-extra}
\usepackage{fix-cm}
\usepackage{alltt}
\usepackage{indentfirst}
\usepackage{amsmath, amsthm, amssymb}
\usepackage{url}

\defaultfontfeatures{Mapping=tex-text}

\setmainfont{Liberation Serif}
\setsansfont{Liberation Sans}
\setmonofont{Liberation Mono}

\setmainlanguage{ukrainian}
\setotherlanguage{english}

\setstretch{1.1}
\begin{document}
\pretolerance=-1
\tolerance=2300

\pagenumbering{arabic}
\pagestyle{empty}
\setlength{\parindent}{1.5cm}
\fontsize{14pt}{6mm}\selectfont

\begin{center}
  Міністерство освіти і науки, молоді та спорту України
  
  Львівський національний університет імені Івана Франка

  Механіко-математичний факультет
\end{center}

\vspace{1cm}

\begin{flushright}
  Кафедра математичного і функціонального аналіза
\end{flushright}

\vspace{4cm}

\begin{center}
  {\bfseries\Large Властивості локсодромних функцій}
\end{center}

\vspace{2cm}

\begin{small}
\begin{flushleft}\leftskip8.5cm
  Магістерська робота студентки 
  групи МТМ-52м\\
  Московчук Наталі\linebreak
  
  Науковий керівник:\\
  професор\\
  Кондратюк А.А.
\end{flushleft}
\end{small}

\vspace{4cm}

\begin{center}
  Львів - 2013 
\end{center}

\clearpage


\setstretch{1.5}
\fontsize{14pt}{6mm}\selectfont

\tableofcontents
\clearpage
\pagestyle{plain}

\section{Основні відомості}

Теорія мероморфних мультиплікативно періодичних функцій була розроблена О.Розенбергом. Ж.Валірон називає ці функції локсодромними, бо точки, в яких такі функції в випадку не дійсного $q$ набувають однакові значення лежaть на логарифмічних спіралях. Образи цих точок на сфері Рімана перетинають меридіани під тим же кутом, і називаються локсодромними кривими ($\lambda o \xi o\zeta$-косий, $\delta \rho o \mu o\zeta$-шлях). В полярних координатах вони прямі лінії.\\
(Двічі періодичні мероморфні функції є еліптичними функціями і є більш відомими з робіт К.Якобі, Н.Абеля, К.Вейєрштрасса.\\
Локсодромні мероморфні функції дають просту конструкцію еліптичних функцій.)\\
В цій роботі вивчаються властивості локсодромних функцій.
\[\begin{array}{l}
\end{array} \]

\newtheorem{ozn}{Означення}
\begin{ozn}
 Мероморфна функція $f$ в проколеній площині $\mathbb{C}^{*}$ називається локсодромною з мультиплікатором $q$, якщо задовільняється умова 
\begin{equation}\label{eq0}
\begin{array}{l}
f(qz)=f(z), 0<\vert q \vert<1, z\in \mathbb{C}^{*}
\end{array}
\end{equation}
\end{ozn}
\[\begin{array}{l}
\end{array} \]

\newtheorem{pryk}{Приклад}
\begin{pryk}
\textit{\textbf{Функція $\mathbf{\rho}$.}}
\begin{ozn}

\end{ozn}

\end{pryk}

\clearpage


\section{Властивості локсодромних функцій}

\newtheorem{twerd}{Твердження}
\begin{twerd}
 Локсодромні функції з мультиплікатором \textit{q} утворюють поле $L_{q}$.
 \end{twerd} 
Поле — це комутативне кільце з одиницею, в якому кожний ненульовий елемент $a\neq 0$ має обернений $a^{-1}$ в цьому ж кільці.\\
Введемо операції додавання і множення.Вони вводяться поточково:\\
$f+g=h, \forall z \in \mathbb{C}^{*} h(z)=f(z)+g(z)\\
f*g=h, \forall z \in \mathbb{C}^{*} h(z)=f(z)*g(z)$\\
Нульвим елементом є $f(z)=0$ в $\mathbb{C}$.\\
Одиничним елементом є $f(z)=1$ в $\mathbb{C}$.\\
Тоді всі аксіоми поля будуть виконуватися:\\
1. Існування нульового елемента: $g(z)=0, f(qz)$-мероморфна, $f(qz)+g(z)=f(qz)$.\\
2. Додавання мероморфних функцій:$f_{1}(qz),f_{2}(qz)$-мероморфні і локсодромні, сума мероморфних- мероморфна, тому $f_{1}(qz)+f_{2}(qz)$-локсодромна \\
3. Частка. Потрібно показати наступне $f_{1}(qz)/f_{2}(qz)=F(qz)$\\
$f_{1}(qz)$, $f_{2}(qz)$ - мероморфні, частка двох мероморфних - мероморфна.Тому, оскільки $f_{1}(qz)$, $f_{2}(qz)$ -локсодромні, то $ F(qz)$ також локсодромна.\\
\emph{Зауважимо:} $f(z)$-локсодромна, $c \neq 0$. Тоді  $f(cqz)=f(q*cz)=f(cz)$
\[\begin{array}{l}
\end{array} \]

\newtheorem{zauv}{Зауваження}
\begin{zauv} 
Будь-яка функція 
\begin{equation}\label{eq1}
\begin{array}{l}
g(z)=f(e^{2iz\pi/\omega_{1}})
\end{array}
\end{equation}
є періодичною з періодом $\omega_{1}$.
\end{zauv}
Переконаємось, що це зауваження вірне: 
\[\begin{array}{l}
g(  z+\omega_{1}  ) = f(  e^{  2i\pi (z+\omega_{1})  / \omega_{1}  }  ) =  f(  e^{  2iz\pi / \omega_{1}  +2i\pi } ) = f(  e^{  2iz\pi / \omega_{1}  } ) = g(z) 
\end{array}\]
\[\begin{array}{l}
\end{array} \]

\begin{twerd}
 Якщо $f$  -локсодромна з мультиплікатором $q$, то  
\[\begin{array}{l}
g(z)=f(e^{2iz\pi/\omega_{1}})
\end{array}\] 
є подвійно періодичною, a $q=e^{2i\pi \omega_{2}/ \omega_{1}}$, $q<1$, $Im(\omega_{2}/\omega_{1})>0$
\end{twerd}
\begin{proof}
\[\begin{array}{l}
\Lambda = \mathbb{Z}\omega_{1} + \mathbb{Z}\omega_{2} 
\end{array}\]

\[\begin{array}{l}
g(z+m\omega_{1}+ n\omega_{2})= f(e^{2i\pi(z+m\omega_{1}+ n\omega_{2})/\omega_{1}})=\\
=f( e^{2iz\pi}e^{2im\pi}e^{2in\pi\omega_{2} / \omega_{1}})=
f(e^{2iz\pi/\omega_{1}}q^{n})=
\end{array}\]
 оскільки $q$-мультиплікатор, то
 \[\begin{array}{l}
 =f(e^{2iz\pi/\omega_{1}})=g(z)
 \end{array}\]
\end{proof}
\[\begin{array}{l}
\end{array} \]

\newtheorem{thm}{Теорема}
\begin{thm}
Якщо f локсодромна і голоморфна в $\mathbb{C}^{*}$, то $f=const$.
\end{thm}

\includegraphics[scale=0.9]{th1.png}
\begin{proof}
Оскільки $f$ локсодромна, тому вона визначається своїми значеннями в $\overline{C_{q}(1)}$ тому, що 
\begin{equation}\label{Cond1}
\begin{array}{l}
 \forall z $ $\exists n: q^{n}z \in \overline{C_{q}(1)} $ ($f(zq^{n})=f(z)$)$ 
\end{array}
\end{equation} 
( умова $\eqref{Cond1}$ рівносильна такій: $f(zq^{n})=f(z)$)\\
тобто
\begin{equation}\label{Cond2}
\begin{array}{l}
\vert q \vert \leq \vert q\vert ^{n} \vert z\vert \leq 1                       
\end{array}
\end{equation}
Щоб довести, що $\eqref{Cond1}$ справді виконується, нам достатньо знайти $n$. З $\eqref{Cond2}$ отримаємо:
\[\begin{array}{l}
log\vert q \vert \leq n log\vert q\vert +log \vert z\vert \leq 0                       
\end{array}\]

\[\begin{array}{l}
\frac {log\vert q \vert }{log\vert q\vert} \geq n  + \frac{ log \vert z\vert}{log\vert q\vert} \geq 0                       
\end{array}\]

\[\begin{array}{l}
1- \frac{ log \vert z\vert}{log\vert q\vert} \geq n  \geq -\frac{ log \vert z\vert}{log\vert q\vert} 
\end{array}\]
Оскільки $\overline{C_{q}(1)}$-компакт, то $f$ компактна на $\overline{C_{q}(1)}$. Зважаючи на теорему: голоморфна на компакті функція обмежена за модулем( бо неперервна за теоремою Вейєрштрасса), отримуємо
\[\begin{array}{l}
  \vert f(z) \vert\leq M
\end{array}\]
Тобто $f$ компактна в $\mathbb{C}^{*}$. \\
А оскільки $f$ компактна в проколеній площині $\mathbb{C}^{*}$, то вона обмежена в околі нуля. Отже, $0$- усувна точка і $f$ по неперервності продовжується
\[\begin{array}{l}
  f(0)=\lim_{z\rightarrow 0} f(z)
\end{array}\]
Тому 
\[\begin{array}{l}
  \vert f(z) \vert\leq M \Longrightarrow \vert f(0) \vert\leq M
\end{array}\]
Отож, $f$ ціла і обмежена в $\mathbb{C}$. Згідно з теоремою Ліувілля, ціла функція, модуль якої обмежений, є сталою.
\end{proof}
\[\begin{array}{l}
\end{array} \]

(Оскільки функція голоморфна, тому вона розвивається в ряд Тейлора і тому: $\exists\lim_{z \to 0}f(z)=A\leq C$, де $C$ деяка константа. Довизначивши функцію $f$: $f(0)=A$, ми отримаємо, що $f$ ціла, оскільки вона аналітична у всій площині. $f$ обмежена в $\overline{C_{q}(1)}$ і в околі нуля $U(0)$. Оскільки значення функції $f$ в кільці ${z\in \mathbb{C}: \vert q \vert<\vert z \vert \leq 1}$ повторюються, за рахунок різних $n$ в $q^{n}$, на $C_{q}(R)={z\in \mathbb{C}: \vert q \vert R < \vert z \vert \leq R}$   і $\vert f \vert \leq C$ в $\mathbb{C}^{*}$. Оскільки $\overline{C_{q}(R)}$- компакт, то $f$- компактна на $\overline{C_{q}(R)}$, тобто і в $\mathbb{C}^{*}$.Теорема з комплексного аналізу стверджує, що $f$ голоморфна в нулі, якщо $f$ є голоморфна і обмежена в околі нуля.Тому ми можемо застосувати теорему Ліувілля \\
(теорема Ліувілля: якщо ціла ф-ція  $f$ обмежена в околі точки $ z=\infty $, то $f(z)=const$.))\\

(Рівність $f(q\xi)=f(\xi)$ для будь-якого $\xi \in \mathbb{C}^{*}$ показує, що $f$ визначається цим обмеженням на кільце $\overline{C_{q}(R)}={z in \mathbb{C}; \vert q \vert R\leq\vert z \omega\leq R }$ Оскільки $\overline{C_{q}(R)}$-компакт, то $f$ компактна на $\overline{C_{q}(R)}$, тобто і в $\mathbb{C}^{*}$. Теорема з комплексного аналізу стверджує, що $f$ голоморфна в нулі, оскільки $f$ є голоморфна і обмежена в околі нуля.Тому ми можемо застосувати теорему Ліувілля )
\[\begin{array}{l}
\end{array} \]
  
\begin{thm}
 Нехай $f\in L_{q}$ і не має ні нулів, ні полюсів на межі кільця 
\begin{equation}\label{th21}
\begin{array}{l}
C_{q}(R)=\lbrace z\in \mathbb{C}:\left |q  \right |R\leq \left |z  \right |<R \rbrace
\end{array}
\end{equation} 
 Тоді сума лишків $\frac{f(z)}{z}$ (лежить в $C_{q}(R)$) дорівнює $0$.
\end{thm}
\begin{proof}
 Межа кільця $C_{q}(R)$ має вигляд $\lbrace z\in \mathbb{C}:\left |q  \right |R< \left |z  \right |<R \rbrace$, адже межею є відкрита множина, але задля спрощення записів, розглядатимемо суму лишків на $C_{q}(R)$, маючи на увазі його межу. Згідно з теоремою Коші про лишки
 \[\begin{array}{l}
\sum_{b_{j}\in C_{q}(R)}^{ } resf(z)= \frac{1}{2i\pi}\int_{\partial C_{q}(R)}^{ } \frac{f(z)}{z}dz
\end{array} \]
Нехай $\Gamma$, $\Gamma^{'}$ позначають кола, які обмежують $C_{q}(R)$.\\
\includegraphics[scale=0.5]{1..PNG}\\
Тоді 
\[\begin{array}{l}
\sum_{b_{j} \in C_{q}(R)}^{ } resf(z) = 
\end{array} \]
оскільки правильним напрямком обходу області інтегрування називається такий
напрямок, коли область інтегрування залишається ліворуч 
\[\begin{array}{l}
= \frac{1}{2i\pi} (\int_{\Gamma^{'}_{+}}^{ }\frac{f(\xi)}{\xi}d\xi  + \int_{\Gamma_{-}}^{ }\frac{f(x)}{x}dx)=
\end{array} \]
оскільки інтеграл по правильному напрямку обходу області інтегрування рівний зі знаком "мінус" інтегралу по протилежному напрямку обходу області інтегрування\\
( тобто $\int_{\Gamma_{-}}^{ }= - \int_{\Gamma_{+}}^{ }$), тому 
\[\begin{array}{l}
=\frac{1}{2\pi}(\int_{\Gamma^{'}_{+}}^{ } \frac{f(\xi)}{\xi}d\xi - \int_{\Gamma_{+}}^{ }\frac{f(x)}{x})dx)=
\end{array} \]
 Записавши межі інтегрування в параметричному вигляді, отримаємо 
 \[\begin{array}{l}
 = \frac{1}{2i\pi}(\int_{0}^{2\pi }f(Re^{i\theta})\frac{iRe^{i\theta}}{Re^{i\theta }}d\theta  + \int_{0}^{2\pi }f(\left |q \right |Re^{i\theta })\frac{i\left |q \right |Re^{i\theta }}{\left |q \right |Re^{i\theta }}d\theta)=
\end{array} \] 
\[\begin{array}{l}
= \frac{1}{2\pi}(\int_{0}^{2\pi} f(Re^{i\theta})d\theta  - \int_{0}^{2\pi}f(\left |q \right |Re^{i\theta })d\theta) = 
\end{array} \]
Зробимо наступні заміни змінних:
\begin{equation}\label{th22}
\begin{array}{l}
\theta =\varphi +\alpha 
\end{array} 
\end{equation}
\begin{equation}\label{th23}
\begin{array}{l}
q=\left |q  \right |e^{i\alpha }
\end{array} 
\end{equation}
і отримаємо
\[\begin{array}{l}
= \frac{1}{2\pi}(\int_{0}^{2\pi} f(Re^{i\theta})d\theta  - \int_{-\alpha}^{2\pi-\alpha}f(qRe^{i\varphi })d\varphi)=
\end{array} \]
а змінивши межі інтегрування другого інтеграла ($2\pi-\alpha->2\pi$, $ -\alpha->0$, $ \\
f(Re^{i\varphi})->f(Re^{i\theta})$, згідно $\eqref {th22} $)
\[\begin{array}{l}
=\frac{1}{2\pi}(\int_{0}^{2\pi} f(Re^{i\theta})d\theta - \int_{0}^{2\pi}f(Re^{i\theta })d\theta)=0,
\end{array}\]
що і потрібно було довести.
\end{proof}
\[\begin{array}{l}
\end{array} \]

\newtheorem{nasl}{Наслідок}
\begin{nasl}
 Кожна не стала локсодромна функція з мультиплікатором $q$ має що найменше 2 полюси( і 2 нулі ) в кожному кільці $C_{q}(R)$.
\end{nasl}
\begin{proof}
$f \neq const $, тому $f$ не голоморфна.(Бо  комплексна функція $u+i*v=f(x+i*y)$ є голоморфною тоді і тільки тоді, коли виконуються умови Коші — Рімана
$\frac{\partial u}{\partial x}=\frac{\partial v}{\partial y};
\frac{\partial u}{\partial y}=-\frac{\partial v}{\partial x}$ )(Ненульові лишки лише в полюсах)??    Сума лишків в полюсах( у всіх ізольованих особливих точках) рівна нулю. ???? Доводимо від супротивного- полюсів $0$ або $1$.\\
Якщо нема жодного полюса, то $f$ голоморфна і стала за попередньою теоремою. \\
Якщо полюс один, то оскільки лишок -це $-1$ коефіцієнт розвинення функції в ряд Лорана, тобто $c_{-1}$, то він за попередньою теоремою рівний нулю.Отримали суперечність.\\
Тобто, справді кожна не стала локсодромна функція має щонайменше 2 полюси в кожному кільці $C_{q}(R)$.
\end{proof}
\[\begin{array}{l}
\end{array} \]

\begin{thm}
Нехай $f\in L_{q}$-кільце, таке як в теоремі 2. Нехай $m_{a}(n_{b})$ позначають кількість нулів(полюсів) $f \in C_{q}(R)$. Тоді 
\begin{center}
$\sum m_{a}=\sum n_{b}$
\end{center}
\end{thm}
\begin{proof}

\begin{equation}\label{arg}
\begin{array}{l}                    
(N-P=\frac{1}{2\pi} \Delta_{\partial G} Arg f(z) =i\sum_{z_k\in G}^{ } res \frac{f^{'}(z)}{f(z)} = \frac{1}{2i\pi}\int_{\partial G}^{ }\frac{f^{'}(z)}{f(z)} dz) 
\end{array}
\end{equation}
Застосуємо принцип аргумента  $\eqref {arg} $
\[\begin{array}{l}
\sum m_{a}-\sum n_{b} = \frac{1}{2i\pi}(\int_{\Gamma^{'}_{+}}^{ } \frac{f^{'}(\xi )}{f(\xi) }d\xi-\int_{\Gamma_{+}}^{ } \frac{f^{'}(z)}{f(z)}dz)=
\end{array}\]
та міркування(виконаємо дії) аналогічні до міркувань в теоремі 2
\[\begin{array}{l}
= \frac{1}{2i\pi}(\int_{0}^{2\pi }iqRe^{i\varphi}\frac{f^{'}(qRe^{\varphi})}{f(qRe^{i\varphi })}d\varphi -
    \int_{0}^{2\pi }iRe^{i\varphi}\frac{f^{'}(Re^{\varphi})}{f(Re^{i\varphi })}d\varphi)=
\end{array}\]
\[\begin{array}{l}
= \frac{1}{2i\pi}(\int_{0}^{2\pi }\frac{1}{q}iqRe^{i\varphi}\frac{f^{'}(Re^{\varphi})}{f(Re^{i\varphi })}d\varphi -
    \int_{0}^{2\pi }iRe^{i\varphi}\frac{f^{'}(Re^{\varphi})}{f(Re^{i\varphi })}d\varphi)=0.
\end{array}\]
\end{proof}
\[\begin{array}{l}
\end{array} \]



2 спосіб.
\[\begin{array}{l}
\sum m_{a}-\sum n_{b} = \frac{1}{2i\pi}(\int_{\Gamma^{'}_{+}}^{ } \frac{f^{'}(\xi )}{f(\xi) }d\xi-\int_{\Gamma_{+}}^{ } \frac{f^{'}(z)}{f(z)}dz)= \\
$ зробивши заміну $z=q\xi$  та перейшовши від $\Gamma$ до $\Gamma^{'}$, отримаємо $\\
= \frac{1}{2i\pi}(\int_{\Gamma^{'}_{+}}^{ } \frac{f^{'}(\xi )}{f(\xi) }d\xi - \int_{\Gamma^{'}_{+}}^{ } \frac{f^{'}(q\xi )}{f(q\xi) }d(q\xi)) = \frac{1}{2i\pi}(\int_{\Gamma^{'}_{+}}^{ } \frac{f^{'}(\xi )}{f(\xi) }d\xi - \int_{\Gamma^{'}_{+}}^{ } \frac{f^{'}(\xi )}{qf(\xi) }qd\xi)=0 \\
$ оскільки $ f(q\xi)=f(\xi)$, $qf^{'}(q\xi)=f^{'}(\xi).
\end{array}\]
\[\begin{array}{l}
\end{array} \]

\begin{twerd} Для $z \in \mathbb{Z}^{*}$ покладемо
\[\begin{array}{l} 
 S(z)= \prod_{0}^{+\infty } ( 1-q^{n}z ) \prod_{1}^{+\infty }( 1-\frac{q^{n}}{z})
 \end{array}\]
Тоді
\begin{equation}\label{s1}
\begin{array}{l}
   S(qz)=-\frac{1}{z}S(z)
 \end{array}
\end{equation}  
\begin{equation}\label{s2}
\begin{array}{l}        
   S(\frac{1}{z})= -\frac{1}{z}S(z)
\end{array}
\end{equation}
\end{twerd}
\begin{proof}
\[\begin{array}{l} 
S(qz)= \prod_{0}^{+\infty } ( 1-q^{n+1}z ) \prod_{1}^{+\infty }( 1-\frac{q^{n}}{qz})=
 \prod_{0}^{+\infty } ( 1-qq^{n}z ) \prod_{1}^{+\infty }( 1-\frac{1}{q}\frac{q^{n}}{z})=
 \end{array}\]
зробимо заміни: для першого добутку $n+1=m$, для другого $n-1=m$, тоді 
\[\begin{array}{l} 
 =\prod_{1}^{+\infty } ( 1-q^{m}z ) \prod_{0}^{+\infty }( 1-\frac{q^{m}}{z})=\frac{\prod_{0}^{+\infty } ( 1-q^{m}z )}{1-z}(1-\frac{1}{z})\prod_{1}^{+\infty }( 1-\frac{q^{m}}{z})=-z^{-1}S(z)
\end{array}\]
\[\begin{array}{l} 
 S(\frac{1}{z})= \prod_{0}^{+\infty } ( 1-q^{n}\frac{1}{z} ) \prod_{1}^{+\infty }( 1-q^{n}z)=
\end{array}\]
домножимо перший добуток на $(1-q^{0}\frac{1}{z})$, а другий на $(1-z)$, тоді 
\[\begin{array}{l} 
  =\prod_{1}^{+\infty } ( 1-q^{n}\frac{1}{z} )(1-q^{0}\frac{1}{z})\frac{    \prod_{0}^{+\infty }( 1-q^{n}z)}{1-z}=-z^{-1}S(z)
 \end{array}\]
\end{proof}
\[\begin{array}{l}
\end{array} \]

Тепер розглянемо $a_{1},...,a_{m},b_{1},...,b_{m}\in \mathbb{C}^{*}$ такі що\\
\begin{equation}\label{w1}
\begin{array}{l} 
 a_{i}\neq b_{j}$  i  $a_{1}*...*a_{m}=b_{1}*...*b_{m}
\end{array}
\end{equation} 

Якщо для $z \in \mathbb{C}^{*}$ не рівному $b_{1},...,b_{m}$ по модулю $\left \langle q \right \rangle $ ми покладемо
\begin{center}
\[ M(z)=\frac{S(\frac{z}{a_{1}})...S(\frac{z}{a_{m}})}{S(\frac{z}{b_{1}})...S(\frac{z}{b_{m}})}\]
\end{center}
тоді бачимо, що $M$ мероморфна функція на $\mathbb{C}^{*}$, її полюси конгруентні $b_{1},...,b_{m}\mod \left \langle q \right \rangle $. Більше того, з $\eqref{s1}$ і $\eqref{s2}$ випливає

\begin{equation}\label{q2}
\begin{array}{l} 
 M(qz)=M(z)$,  $M(\frac{1}{z})=\frac{S(a_{1}z)...S(a_{m}z)}{S(b_{1}z)...S(b_{m}z)}     
\end{array}
\end{equation}  
 
тобто зокрема ми бачимо, що $M\in L_{q}$(локсодромна).\\
Перевіримо $\eqref{q2}$:\\  
 \[\begin{array}{l}
 M(qz)=\frac{S(\frac{qz}{a_{1}})...S(\frac{qz}{a_{m}})}{S(\frac{qz}{b_{1}})...S(\frac{qz}{b_{m}})}=
 \end{array}\]
 застосуємо рівність $\eqref{s1}$
\[\begin{array}{l}
=\frac{ (-1)^{m} \frac{a_{1}}{z} S(\frac{z}{a_{1}})...\frac{a_{m}}{z} S(\frac{z}{a_{m}}) }{ (-1)^{m} \frac{b_{1}}{z} S(\frac{z}{b_{1}})...\frac{b_{m}}{z} S(\frac{z}{b_{m}}) }

=\frac{ a_{1}*...*a_{m}}{b_{1}*...*b_{m}}\frac{S(\frac{z}{a_{1}})...S(\frac{z}{a_{m}})}{S(\frac{z}{b_{1}})...S(\frac{z}{b_{m}})}=
\end{array}\]
\[\begin{array}{l}
=\frac{ a_{1}*...*a_{m}}{b_{1}*...*b_{m}}M(z) 
\end{array}\]
зважаючи на $\eqref{w1}$, отримаємо
$M(qz)=M(z)$\\

Аналогічно застосувавши $\eqref{s2}$ і $\eqref{w1}$, отримаємо\\
\[\begin{array}{l}
M(\frac{1}{z})=\frac{S(\frac{1}{za_{1}})...S(\frac{1}{za_{m}})}{S(\frac{1}{zb_{1}})...S(\frac{1}{zb_{m}})}= 
\frac{(-1)^m \frac{1}{za_{1}} S(za_{1})...\frac{1}{za_{m}} S(za_{m})}{(-1)^m \frac{1}{zb_{1}} S(zb_{1})...\frac{1}{zb_{m}} S(zb_{m})}=
\end{array}\]
\[\begin{array}{l}
= \frac{b_{1}...b_{m}}{a_{1}...a_{m}}
\frac{ S(za_{1})...S(za_{m})}{S(zb_{1})...S(zb_{m})}=\frac{ S(za_{1})...S(za_{m})}{S(zb_{1})...S(zb_{m})}  
\end{array} \]
\[\begin{array}{l}
\end{array} \]

\begin{thm}
Для кожного $R$ на межі $C_{q}(R)$, яка не містить ні нулів, ні полюсів локсодромної функції $f\in L_{q}$, нехай $\lambda=\frac{(a_{1}...a_{m})}{(b_{1}...b_{m})}$ , де $a_{1},...,a_{m}$- нулі $f\in L_{q}$, $b_{1},...,b_{m}$- полюси $f\in L_{q}$. Тоді $\lambda\in \left \langle q \right \rangle$. ($ \exists n:\lambda =q^{n}$)
\end{thm}
\begin{proof}
Першу рівність з $\eqref{q2}$ перепишемо:
\[\begin{array}{l}
 M(qz)=\lambda M(z).
 \end{array}\]
 Розглянемо функцію:
 \[\begin{array}{l}
 g(z)=\frac{f(z)}{M(z)}.
  \end{array}\]
  Згідно побудови $M$,бачимо, що $g$ не має ні нулів, ні полюсів в $C_{q}(R)$ (бо $M$ має ті ж нулі і полюси, що й $f$). Звідси слідує, що $g$ ціла так само, як $\frac{1}{g}$.
\begin{equation}\label{th41}
\begin{array}{l}
g(qz)=\frac{f(qz)}{M(qz)}=\frac{f(z)}{\lambda M(z)}=\frac{1}{\lambda}g(z)
\end{array}
\end{equation}
Нехай $\sum_{-\infty}^{+\infty}c_{n}z^{n}$-розвинення в ряд Лорана функції $g$ в $\mathbb{C}^{*}.\\$
 \[\begin{array}{l}
g(z)=\sum_{-\infty}^{+\infty}c_{n}z^{n},
 \end{array}\]
 \[\begin{array}{l}
 g(qz)=\sum_{-\infty}^{+\infty}c_{n}q^{n}z^{n}=\frac{1}{\lambda}g(z)
  \end{array}\]
З огляду на $\eqref {th41} $ і того, що хоча б один з $c_{n}$ ($\exists k: c_{k}\neq 0$) відмінний від нуля, ми бачимо, що 
 \[\begin{array}{l}
 (\lambda q^{n}-1)c_{n}=0,
  \end{array}\]
  отже 
 \[\begin{array}{l} 
 \lambda= q^{-n}\in \left \langle q \right \rangle .
 \end{array}\]
Розвинення в ряд Лорана єдине, тому коефіцієнти при відповідних степенях рівні.
\[\begin{array}{l}
c_{k}q^{k}=\frac{1}{\lambda} c_{k}$ $\Rightarrow$ $(q^{k}\lambda -1)c_{k}=0,
 \end{array}\]
\[\begin{array}{l}
c_{k}\neq 0$ $\Rightarrow$  $\lambda=q^{-k}.
 \end{array}\]
??????Для $g(z)$ лише одне $ c_{k}\neq 0$. ????
\end{proof}
\[\begin{array}{l}
\end{array} \]

\begin{thm}
Якщо $\lambda =q^{n}$, $f$- локсодромна функція з теореми 4, то  $f$ має вигляд
\begin{equation}\label{th51}
\begin{array}{l}
 f(z)= C\tfrac{S(\frac{z}{a_{1}})...S(\frac{z}{a_{m}})}{S(\frac{zq^{n}}{b_{1}})S(\frac{z}{b_{2}})...S(\frac{z}{b_{m}})}
\end{array}
\end{equation}
 \end{thm}
\begin{proof}
\[\begin{array}{l}
g(z)=\frac{f(z)}{M(z)}\Rightarrow f(z)=M(z)g(z)= M(z)z^{n}c_{n}
 \end{array}\]
 позначивши $c_{n}=C$, отримаємо 
 \begin{equation}\label{th52}
\begin{array}{l} 
f(z) =Cz^{n}\frac{S(\frac{z}{a_{1}})...S(\frac{z}{a_{m}})}{S(\frac{z}{b_{1}})...S(\frac{z}{b_{m}})} 
\end{array}
  \end{equation}
\[\begin{array}{l}
f(z)=C\tfrac{S(\frac{z}{a_{1}})...S(\frac{z}{a_{m}})}{S(\frac{zq^{n}}{b_{1}})S(\frac{z}{b_{2}})...S(\frac{z}{b_{m}})}= 
 \end{array}\]
оскільки
\[\begin{array}{l}
S(\frac{zq^{n}}{b_{1}})= -\frac{b_{1}}{z}S(\frac{zq^{n-1}}{b_{1}})=(-1)^{n}\frac{b_{1}^{n}}{z^{n}}S(\frac{z}{b_{1}}).
 \end{array}\]
тому
\[\begin{array}{l}
 =C\tfrac{S(\frac{z}{a_{1}})...S(\frac{z}{a_{m}})}{(-1)^{n}\frac{b_{1}^{n}}{z^{n}} S(\frac{z}{b_{1}})S(\frac{z}{b_{2}})...S(\frac{z}{b_{m}})}=Cz^{n}\frac{S(\frac{z}{a_{1}})...S(\frac{z}{a_{m}})}{S(\frac{z}{b_{1}})...S(\frac{z}{b_{m}})} 
  \end{array}\]
А константа $C$ дорівнює $(-1)^{n}b_{1}^{-n}$.
\end{proof}


\clearpage
\section{Приклади}
І так далі\cite{web}



\clearpage
\addcontentsline{toc}{section}{Література}
\begin{thebibliography}{9}

  \bibitem{alias}<Yves Hellegouarch> \emph{<Invitation to the mathematics of Fermat-Wiles>},
    <2001>, <385> ст.
    
  \bibitem{alias}<Гольдберг А.А., Шеремета М.М.> \emph{<Аналітичні функції>},
    <Київ НМК ВО> <1991>, <115> ст.
      
  \bibitem{web}<АВТОР> \emph{<НАЗВА>} [Електронний ресурс],
    <РІК>. Режим доступу:
    \url{https://github.com/Uko/thesis-template}

\end{thebibliography}

\end{document}
