\documentclass[12pt,a4paper]{article}

\usepackage{fontspec}
\usepackage{polyglossia}
\usepackage[left=2.5cm,top=2.5cm,right=2.5cm,bottom=2.5cm,nohead]{geometry}
\usepackage{setspace}
\usepackage{listings}
\usepackage{color}
\usepackage{float}
\usepackage{courier}
\usepackage{bold-extra}
\usepackage{fix-cm}
\usepackage{alltt}
\usepackage{indentfirst}
\usepackage{amsmath, amsthm, amssymb}
\usepackage{url}

\defaultfontfeatures{Mapping=tex-text}

\setmainfont{Liberation Serif}
\setsansfont{Liberation Sans}
\setmonofont{Liberation Mono}

\setmainlanguage{ukrainian}
\setotherlanguage{english}

\setstretch{1.1}

\begin{document}
\pretolerance=-1
\tolerance=2300

\thispagestyle{empty}
\setlength{\parindent}{1.5cm}
\fontsize{14pt}{6mm}\selectfont

\begin{center}
  Міністерство освіти і науки, молоді та спорту України
  
  Львівський національний університет імені Івана Франка

  Механіко-математичний факультет
\end{center}

\vspace{1cm}

\begin{flushright}
  Кафедра математичного і функціонального аналіза
\end{flushright}

\vspace{4cm}

\begin{center}
  {\bfseries\Large Властивості локсодромних функцій}
\end{center}

\vspace{2cm}

\begin{small}
\begin{flushleft}\leftskip8.5cm
  Магістерська робота студентки 
  групи МТМ-52м\\
  Московчук Наталі\linebreak
  
  Науковий керівник:\\
  професор\\
  Кондратюк А.А.
\end{flushleft}
\end{small}

\vspace{4cm}

\begin{center}
  Львів - 2013 
\end{center}

\clearpage



\setstretch{1.5}
\fontsize{14pt}{6mm}\selectfont

\newcommand{\vect}[1]{(#1_1,#1_2,#1_3,\dots,#1_n)}

\thispagestyle{empty}
\tableofcontents
\clearpage
\pagenumbering{arabic}

\section{<Основні відомості\#>}

Теорія мероморфних мультиплікативно періодичних функцій була розроблена О.Розенбергом. Ж.Валірон називає ці функції локсодромними, бо точки, в яких такі функції в випадку ,,,,,,,не дійсного $q$ набуває такого ж значення ??? лежить на логарифмічній спіралі. Образи цих ??останніх.точок?? на сфері Рімана перетинають меридіани під тим же кутом, і називаються локсодромними кривими ($\lambda o \xi o\zeta$-косий, $\delta \rho o \mu o\zeta$-шлях). В лог????-полярних координатах вони прямі лінії.\\
Двічі періодичні мероморфні функції є еліптичними функціями і є більш відомими з робіт К.Якобі, Н.Абеля, К.Вейєрштрасса.\\
Локсодромні мероморфні функції дають просту конструкцію еліптичних функцій.\\

\emph{Означення:} Мероморфна функція $f$ в проколеній площині $\mathbb{C}^{*}$ називається локсодромною з мультиплікатором $q$, якщо задовільняється умова \begin{center}
$f(qz)=f(z), 0<\vert q \vert<1, z\in \mathbb{C}^{*}.$
\end{center}

\clearpage

\section{<Властивості локсодромних функцій\#>}

\emph{Твердження1:} Локсодромні функції з мультиплікатором \textit{q} утворюють поле $L_{q}$. \\
Поле — це комутативне кільце з одиницею, в якому кожний ненульовий елемент $a\neq 0$ має обернений $a^{-1}$ в цьому ж кільці.\\
Введемо операції додавання і множення.Вони вводяться поточково:\\
$f+g=h, \forall z \in \mathbb{C}^{*} h(z)=f(z)+g(z)\\
f*g=h, \forall z \in \mathbb{C}^{*} h(z)=f(z)*g(z)$\\
Нульвим елементом є $f(z)=0 в \mathbb{C}$.\\
Одиничним елементом є $f(z)=1 в \mathbb{C}$.\\
Тоді всі аксіоми поля будуть виконуватися:\\
1. Існування нульового елемента: $g(z)=0, f(qz)$-мероморфна, $f(qz)+g(z)=f(qz)$.\\
2. Додавання мероморфних функцій:$f_{1}(qz),f_{2}(qz)$-мероморфні і локсодромні, сума мероморфних- мероморфна, тому $f_{1}(qz)+f_{2}(qz)$-локсодромна \\
3. Частка. Потрібно показати наступне $f_{1}(qz)/f_{2}(qz)=F(qz)$\\
$f_{1}(qz)$, $f_{2}(qz)$ - мероморфні, частка двох мероморфних - мероморфна.Тому, оскільки $f_{1}(qz)$, $f_{2}(qz)$ -локсодромні, то $ F(qz)$ також локсодромна.\\
4. Композиція.$f(z)$-локсодромна, $f(cz)$-локсодромна, $c \neq 0$. Тоді  $f(cqz)=f(q*cz)=f(cz)$\\

\emph{Зауваження:} Будь-яка $g(z)=f(e^{2iz\pi/\omega_{1}})$ є періодичною з періодом $\omega_{1}$.\\
Перевіримо це зауваження: $g(  z+\omega_{1}  ) = f(  e^{  2i\pi (z+\omega_{1})  / \omega_{1}  }  ) =  f(  e^{  2iz\pi / \omega_{1}  +2i\pi } ) = f(  e^{  2iz\pi / \omega_{1}  } ) = g(z)$ \\

\emph{Твердження2:} Якщо $f$  -локсодромна з мультиплікатором $q$, то  $g(z)=f(e^{2iz\pi/\omega_{1}})$ є подвійно періодичною, a $q=e^{2i\pi \omega_{2}/ \omega_{1}}$, $q<1$, $Im(\omega_{2}/\omega_{1})>0$\\
Доведення: $ \Lambda = \mathbb{Z}\omega_{1} + \mathbb{Z}\omega_{2} $ \\
$g(z+m\omega_{1}+ n\omega_{2})= f(e^{2i\pi(z+m\omega_{1}+ n\omega_{2})/\omega_{1}})=f( e^{2iz\pi}e^{2im\pi}e^{2in\pi\omega_{2} / \omega_{1}} )=f(e^{2iz\pi/\omega_{1}}q^{n})= $, оскільки $q$-мультиплікатор, то
 $=f(e^{2iz\pi/\omega_{1}})=g(z)$\\
 
\emph{Теорема1:}Якщо f локсодромна і голоморфна в $\mathbb{C}^{*}$, то $f=const$.)\\
Доведення:\\
\\
\\

Рівність $f(q\xi)=f(\xi)$ для будь-якого $\xi \in \mathbb{C}^{*}$ показує, що $f$ визначається цим обмеженням на кільце????????????????????????????????????????????????????/ Оскільки???????????????????????-компакт, то $f$ компактна на ????????????????????, тобто і в $\mathbb{C}^{*}$. ????Теорема з комплексного аналізу стверджує, що $f$ голоморфна в нулі ???(origin)????, оскільки $f$ є голоморфна і обмежена в околі ?????(neighbourhood)??? нуля????(origin)????.Тому ми можемо застосувати теорему Ліувілля \\


\emph{Теорема2:} Нехай $f\in L_{q}$ і не має ні нулів, ні полюсів на межі кільця $C_{q}(R)={z\in \mathbb{C}:\left |q  \right |R\leq \left |z  \right |<R }$. Тоді сума лишків $f(z)/z$ (лежить в $C_{q}(R)$) дорівнює $0$.\\
Доведення:\\Згідно з теоремою Коші про лишки\\
\begin{center}
$ \sum_{b_{j}\in C_{q}(R)}^{ } resf(z)= \frac{1}{2i\pi}\int_{\partial C_{q}(R)}^{ } \frac{f(z)}{z}dz$
\end{center}
Нехай $\Gamma$, $\Gamma^{'}$ позначають кола, які обмежують $C_{q}(R)$. ...ГРАФІК............................
\\

Тоді \\
$\sum_{b_{j}\in C_{q}(R)}^{ } resf(z)=
\\=\frac{1}{2i\pi}(\int_{\Gamma^{'}_{+}}^{ } \frac{f(\xi)}{\xi}d\xi  +\int_{\Gamma_{_}}^{ }\frac{f(x)}{x}dx)=   \\
\frac{1}{2\pi}(\int_{\Gamma^{'}_{+}}^{ } \frac{f(\xi)}{\xi} d\xi -\int_{\Gamma_{+}}^{ }\frac{f(x)}{x}})dx)=
\\
=\frac{1}{2i\pi}(\int_{0}^{2\pi } \f(Re^{i\theta})frac{iRe^{i\theta}}{Re^{i\theta }}d\theta  +\int_{0}^{2\pi }\f(\left |q \right |Re^{i\theta })frac{i\left |q \right |Re^{i\theta }}{\left |q \right |Re^{i\theta }}d/theta)=   \\
\frac{1}{2\pi}(\int_{0}^{2\pi} \f(Re^{i\theta})d\theta  -\int_{0}^{2\pi}\f(\left |q \right |Re^{i\theta })d/theta)=
\\
\theta =\varphi +\alpha ;
q=\left |q  \right |e^{i\alpha }

\\
= \frac{1}{2\pi}(\int_{0}^{2\pi} \f(Re^{i\theta})d\theta  -\int_{-\alpha}^{2\pi-\alpha}\f(qRe^{i\varphi })d/varphi)=\frac{1}{2\pi}(\int_{0}^{2\pi} \f(Re^{i\theta})d\theta  -\int_{0}^{2\pi}\f(Re^{i\varphi????\theta })d/varphi???\theta)=0$.\\
\\
\emph{Наслідок:} Кожна не стала локсодромна функція з мультиплікатором $q$ має що найменше 2 полюси( і 2 нулі ) в кожному кільці $C_{q}(R)$.\\
  
\emph{Теорема3:} Нехай $f\in L_{q}$ і $C_{q}(R)$- кільце, таке як в теоремі 2. Нехай $m_{a}$($n_{b}$) позначає кількість нулів (полюсів) $f$ в $C_{q}(R)$. Тоді $\sum m_{a}=\sum n_{b}$ \\
Доведення:
1 спосіб.\\
$\sum m_{a}-\sum n_{b} = \frac{1}{2i\pi}(\int_{\Gamma^{'}_{+}}^{ } \frac{f^{'}(\xi )}{f(\xi) }d\xi-\int_{\Gamma_{+}}^{ } \frac{f^{'}(z)}{f(z)}dz)=$ (зробивши заміну $z=q\xi$  та перейшовши від $\Gamma$ до $\Gamma^{'}$, отримаємо) $= \frac{1}{2i\pi}(\int_{\Gamma^{'}_{+}}^{ } \frac{f^{'}(\xi )}{f(\xi) }d\xi - \int_{\Gamma^{'}_{+}}^{ } \frac{f^{'}(q\xi )}{f(q\xi) }d(q\xi)) = \frac{1}{2i\pi}(\int_{\Gamma^{'}_{+}}^{ } \frac{f^{'}(\xi )}{f(\xi) }d\xi - \int_{\Gamma^{'}_{+}}^{ } \frac{f^{'}(\xi )}{qf(\xi) }qd\xi)=0 $, оскільки $ f(q\xi)=f(\xi)$, $qf^{'}(q\xi)=f^{'}(\xi)$.\\
2 спосіб.\\
$\sum m_{a}-\sum n_{b} = \frac{1}{2i\pi}(\int_{\Gamma^{'}_{+}}^{ } \frac{f^{'}(\xi )}{f(\xi) }d\xi-\int_{\Gamma_{+}}^{ } \frac{f^{'}(z)}{f(z)}dz) = \frac{1}{2i\pi}(\int_{0}^{2\pi} \frac{f^{'}(qRe^{i\phi???})}{f(qRe^{i\phi???}) }iqRe^{i\phi???}d\phi - \int_{0}^{2\pi} \frac{f^{'}(Re^{i\phi???})}{f(Re^{i\phi???}) }iRe^{i\phi???}d\phi) = \frac{1}{2i\pi}(\int_{0}^{2\pi} \frac{f^{'}(Re^{i\phi???})}{f(Re^{i\phi???}) }\frac{1}{q} iqRe^{i\phi???}d\phi - \int_{0}^{2\pi} \frac{f^{'}(Re^{i\phi???})}{f(Re^{i\phi???}) }iRe^{i\phi???}d\phi)=0 $\\

\emph{Твердження4:} Для $z \in \mathbb{Z}^{*}$ покладемо
  \begin{center}
	$S(z)= \prod_{0}^{+\infty } ( 1-q^{n}z ) \prod_{1}^{+\infty }( 1-q^{n}z^{-1} )$
  \end{center} Тоді виконується $S(qz)=-z^{-1}S(z)$ (1), $S(\frac{1}{z}=-z^{-1}S(z)$ (2)\\
Доведення:........................................\\
$S(qz)= \prod_{0}^{+\infty } ( 1-q^{n+1}z ) \prod_{1}^{+\infty }( 1-\frac{q^{n}}{qz})=
 \prod_{0}^{+\infty } ( 1-qq^{n}z ) \prod_{1}^{+\infty }( 1-\frac{1}{q}\frac{q^{n}}{z})=$, зробимо заміни: для першого добутку $n+1=m$, для другого $n-1=m$, тоді $=\prod_{1}^{+\infty } ( 1-q^{m}z ) \prod_{0}^{+\infty }( 1-\frac{q^{m}}{z})=\frac{\prod_{0}^{+\infty } ( 1-q^{m}z )}{1-z}(1-\frac{1}{z})\prod_{1}^{+\infty }( 1-\frac{q^{m}}{z})=-z^{-1}S(z)$ \\
 $S(\frac{1}{z})= \prod_{0}^{+\infty } ( 1-q^{n}\frac{1}{z} ) \prod_{1}^{+\infty }( 1-q^{n}z)=$, домножимо перший добуток на $(1-q^{0}\frac{1}{z})$, а другий на $(1-z)$, тоді $=\prod_{1}^{+\infty } ( 1-q^{n}\frac{1}{z} )(1-q^{0}\frac{1}{z})\frac{ \prod_{0}^{+\infty }( 1-q^{n}z)}{1-z}=-z^{-1}S(z)$\\
 
Розглянемо $a_{1},...,a_{m},b_{1},...,b_{m}\in \mathbb{C}^{*}$ такі що\\
 $a_{i}\neq b_{j}$ i $a_{1}*...*a_{m}=b_{1}*...*b_{m}$.(*)\\

Якщо для $z \in \mathbb{C}^{*}$ не рівному $b_{1},...,b_{m}$ по модулю $\left \langle q \right \rangle $ ми покладемо $M(z)=\frac{S(\frac{z}{a_{1}})...S(\frac{z}{a_{m}})}{S(\frac{z}{b_{1}})...S(\frac{z}{b_{m}})}$,тоді бачимо, що $M$ мероморфна функція на $\mathbb{C}^{*}$, її полюси конгруентні $b_{1},...,b_{m}\mod \left \langle q \right \rangle $. Більше того, з (1) і (2) випливає $M(qz)=M(z)$, $M(\frac{1}{z})=\frac{S(a_{1}z)...S(a_{m}z)}{S(b_{1}z)...S(b_{m}z)}$ (**), тобто зокрема ми бачимо, що $M\in L_{q}$(локсодромна).\\
Перевіримо це:\\
$M(qz)=\frac{S(\frac{qz}{a_{1}})...S(\frac{qz}{a_{m}})}{S(\frac{qz}{b_{1}})...S(\frac{qz}{b_{m}})}$\\
$S(\frac{qz}{a_{j}})= \prod_{0}^{+\infty } ( 1-q^{n}\frac{qz}{a_{j}} ) \prod_{1}^{+\infty }( 1-q^{n}\frac{a_{j}}{qz}) =?????????????????????????????????????= \frac{1}{1-\frac{z}{a_{j}}} \prod_{0}^{+\infty } ( 1-q^{n}\frac{z}{a_{j}}) (1-\frac{a_{j}}{z}) \prod_{1}^{+\infty }( 1-q^{n}\frac{a_{j}}{z})=$
(оскільки $ \frac{\frac{z-a_{j}}{z}}{\frac{a_{j}-z}{a_{j}}}= \frac{(z-a_{j})a_{j}}{-(z-a_{j})z}=-\frac{a_{j}}{z} $, тому)
$=-\frac{a_{j}}{z}S(\frac{z}{a_{j}})$\\
$S(\frac{z}{a_{j}})=  \prod_{0}^{+\infty } ( 1-q^{n}\frac{z}{a_{j}} ) \prod_{1}^{+\infty }( 1-q^{n}\frac{a_{j}}{z})$.\\
 Тодi
$M(qz)=\frac{ (-1)^{m} \frac{a_{1}}{z} S(\frac{z}{a_{1}})...\frac{a_{m}}{z} S(\frac{z}{a_{m}}) }{ (-1)^{m} \frac{b_{1}}{z} S(\frac{z}{b_{1}})...\frac{b_{m}}{z} S(\frac{z}{b_{m}}) }=\frac{ a_{1}*...*a_{m}}{b_{1}*...*b_{m}}M(z)=$ (зважаючи на (*)) $=M(z)$\\

\emph{Теорема4:} Для кожного $R$ на межі $C_{q}(R)$, яка не містить ні нулів, ні полюсів локсодромної функції $f\in L_{q}$, нехай $\lambda=\frac{(a_{1}...a_{m})}{(b_{1}...b_{m})}$ , де $a_{1},...,a_{m}$- нулі $f\in L_{q}$, $b_{1},...,b_{m}$- полюси $f\in L_{q}$. Тоді $\lambda\in \left \langle q \right \rangle$. ($ \exists n:\lambda =q^{n}$)\\
Доведення:\\
Першу рівність з (**) перепишемо:$ M(qz)=\lambda M(z)$. Розглянемо функцію:$ g(z)=\frac{f(z)}{M(z)}$. За побудовою $М$, $g$ не має ні нулів, ні полюсів в $C_{q}(R)$ (бо $M$ має ті ж нулі і полюси, що й $f$). Звідси слідує, що $g$ ціла так само, як $\frac{1}{g}$.\\
$g(qz)=\frac{f(qz)}{M(qz)}=\frac{f(z)}{\lambda M(z)}=\frac{1}{\lambda}g(z)$ (1)\\
Нехай $\sum_{-\infty}^{+\infty}c_{n}z^{n}$-розвинення в ряд Лорана функції $g$ в $\mathbb{C}^{*}$.\\
$g(z)=\sum_{-\infty}^{+\infty}c_{n}z^{n}$, $g(qz)=\sum_{-\infty}^{+\infty}c_{n}q^{n}z^{n}=\frac{1}{\lambda}g(z)$\\
З огляду на (1) і того, що хоча б один з $c_{n}$ ($\exists k: c_{k}\neq 0$) відмінний від нуля, ми бачимо, що $(\lambda q^{n}-1)c_{n}=0$, отже $\lambda= q^{-n}\in \left \langle q \right \rangle $.\\
Розвинення в ряд Лорана єдине, тому коефіцієнти при відповідних степенях рівні.
$c_{k}q^{k}=\frac{1}{\lambda} c_{k}$ $\Rightarrow$ $(q^{k}\lambda -1)c_{k}=0$, $c_{k}\neq 0$ $\Rightarrow$  $\lambda=q^{-k}$.\\
??????Для $g(z)$ лише одне $ c_{k}\neq 0$. ????\\

\emph{Теорема5:}Якщо $\lambda =q^{n}$, $f$- локсодромна функція з теореми 4, то  $f$ має вигляд\\
 \begin{center}
 $f(z)= C\tfrac{S(\frac{z}{a_{1}})...S(\frac{z}{a_{m}})}{S(\frac{zq^{n}}{b_{1}})S(\frac{z}{b_{2}})...S(\frac{z}{b_{m}})}$
 \end{center}
Доведення:\\
$g(z)=\frac{f(z)}{M(z)}\Rightarrow f(z)=M(z)g(z)= M(z)z^{n}c_{n}=$, позначивши $c_{n}=C$, отримаємо $=Cz^{n}\frac{S(\frac{z}{a_{1}})...S(\frac{z}{a_{m}})}{S(\frac{z}{b_{1}})...S(\frac{z}{b_{m}})}$ (2)\\
$f(z)=C\tfrac{S(\frac{z}{a_{1}})...S(\frac{z}{a_{m}})}{S(\frac{zq^{n}}{b_{1}})S(\frac{z}{b_{2}})...S(\frac{z}{b_{m}})}= C\tfrac{S(\frac{z}{a_{1}})...S(\frac{z}{a_{m}})}{(-1)^{n}\frac{b_{1}^{n}}{z^{n}} S(\frac{z}{b_{1}})S(\frac{z}{b_{2}})...S(\frac{z}{b_{m}})}=(2)$, оскільки\\
$S(\frac{zq^{n}}{b_{1}})= -\frac{b_{1}}{z}S(\frac{zq^{n-1}}{b_{1}})=(-1)^{n}\frac{b_{1}^{n}}{z^{n}}S(\frac{z}{b_{1}})$.\\
А константа $C$ дорівнює $(-1)^{n}b_{1}^{-n}$.\\

І так далі\cite{web}

\clearpage
\section{<РОЗДІЛ Основне завдання\#>}



І так далі\cite{web}

\clearpage
\addcontentsline{toc}{section}{Література}
\begin{thebibliography}{9}

  \bibitem{alias}<АВТОР> \emph{<КНИГА>},
    <ВИДАВНИЦТВО> <РІК>, <К-КІСТЬ СТОРІНОК> ст.
    
  \bibitem{web}<АВТОР> \emph{<НАЗВА>} [Електронний ресурс],
    <РІК>. Режим доступу:
    \url{https://github.com/Uko/thesis-template}

\end{thebibliography}

\end{document}
