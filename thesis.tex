\documentclass[12pt,a4paper]{article}

\usepackage{fontspec}
\usepackage{polyglossia}
\usepackage[left=2.5cm,top=2.5cm,right=2.5cm,bottom=2.5cm,nohead]{geometry}
\usepackage{setspace}
\usepackage{listings}
\usepackage{color}
\usepackage{float}
\usepackage{courier}
\usepackage{bold-extra}
\usepackage{fix-cm}
\usepackage{alltt}
\usepackage{indentfirst}
\usepackage{amsmath, amsthm, amssymb}
\usepackage{url}

\setmainfont[Mapping=tex-text]{Times New Roman}
\newfontfamily\cyrillicfont{Times New Roman}
\newfontfamily\cyrillicfonttt{Courier New}
\setmainlanguage{ukrainian}
\setotherlanguage{english}

\setstretch{1.1}

\begin{document}
\pretolerance=-1
\tolerance=2300

\thispagestyle{empty}
\setlength{\parindent}{1.5cm}
\fontsize{14pt}{6mm}\selectfont

\begin{center}
  Міністерство освіти і науки, молоді та спорту України
  
  Львівський національний університет імені Івана Франка

  Механіко-математичний факультет
\end{center}

\vspace{1cm}

\begin{flushright}
  Кафедра математичного і функціонального аналіза
\end{flushright}

\vspace{4cm}

\begin{center}
  {\bfseries\Large Властивості локсодромних функцій}
\end{center}

\vspace{2cm}

\begin{small}
\begin{flushleft}\leftskip8.5cm
  Магістерська робота студентки 
  групи МТМ-52м\\
  Московчук Наталі\linebreak
  
  Науковий керівник:\\
  професор\\
  Кондратюк А.А.
\end{flushleft}
\end{small}

\vspace{4cm}

\begin{center}
  Львів - 2013 
\end{center}

\clearpage



\setstretch{1.5}
\fontsize{14pt}{6mm}\selectfont

\newcommand{\vect}[1]{(#1_1,#1_2,#1_3,\dots,#1_n)}

\thispagestyle{empty}
\tableofcontents
\clearpage
\pagenumbering{arabic}

\section{Вступ}

Лорем іпсум!\cite{alias}

\clearpage

\section{<РОЗДІЛ Реферативна частина\#>}

І так далі\cite{web}

\clearpage
\section{<РОЗДІЛ Основне завдання\#>}

\emph{Твердження1} Локсодромні функції з мультиплікатором \textit{q} утворюють поле $L_{q}$. \\
Перевіримо аксіоми поля:\\
1. Нульовий елемент існує - 0.\\
2. Додавання виконується.\\
3. Частка. Потрібно показати наступне $f_{1}(qz)/f_{2}(qz)=F(qz)$\\
$f_{1}(qz)$, $f_{2}(qz)$ - мероморфні, частка двох мероморфних - мероморфна.Тому, оскільки $f_{1}(qz)$, $f_{2}(qz)$ -локсодромні, то $ F(qz)$ також локсодромна.\\
4. Композиція.$f(z)$-локсодромна, $f(cz)$-локсодромна, $c \neq 0$. Тоді  $f(cqz)=f(q*cz)=f(cz)$\\
\emph{Твердження2} Будь-яка $g(z)=f(e^{2\pi iz/\omega_{1}})$ є періодичною з періодом $\omega_{1}$.\\
$g(  z+\omega_{1}  ) = f(  e^{  2i\pi (z+\omega_{1})  / \omega_{1}  }  ) =  f(  e^{  2iz\pi / \omega_{1}  +2i\pi } ) = f(  e^{  2iz\pi / \omega_{1}  } ) = g(z)$ \\


І так далі\cite{web}

\clearpage
\addcontentsline{toc}{section}{Література}
\begin{thebibliography}{9}

  \bibitem{alias}<АВТОР> \emph{<КНИГА>},
    <ВИДАВНИЦТВО> <РІК>, <К-КІСТЬ СТОРІНОК> ст.
    
  \bibitem{web}<АВТОР> \emph{<НАЗВА>} [Електронний ресурс],
    <РІК>. Режим доступу:
    \url{https://github.com/Uko/thesis-template}

\end{thebibliography}

\end{document}
